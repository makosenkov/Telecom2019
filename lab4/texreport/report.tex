% !TeX spellcheck = ru_RU
\documentclass[a4paper,12pt]{extarticle}
\usepackage[utf8x]{inputenc}
\usepackage[T1,T2A]{fontenc}
\usepackage[russian]{babel}
\usepackage{hyperref}
\usepackage{indentfirst}
\usepackage{listings}
\usepackage{color}
\usepackage{here}
\usepackage{array}
\usepackage{multirow}
\usepackage{graphicx}

\usepackage{caption}
\renewcommand{\lstlistingname}{Программа} % заголовок листингов кода

\bibliographystyle{ugost2008ls}

\usepackage{listings}
\lstset{ %
extendedchars=\true,
keepspaces=true,
language=C,						% choose the language of the code
basicstyle=\footnotesize,		% the size of the fonts that are used for the code
numbers=left,					% where to put the line-numbers
numberstyle=\footnotesize,		% the size of the fonts that are used for the line-numbers
stepnumber=1,					% the step between two line-numbers. If it is 1 each line will be numbered
numbersep=5pt,					% how far the line-numbers are from the code
backgroundcolor=\color{white},	% choose the background color. You must add \usepackage{color}
showspaces=false				% show spaces adding particular underscores
showstringspaces=false,			% underline spaces within strings
showtabs=false,					% show tabs within strings adding particular underscores
frame=single,           		% adds a frame around the code
tabsize=2,						% sets default tabsize to 2 spaces
captionpos=t,					% sets the caption-position to top
breaklines=true,				% sets automatic line breaking
breakatwhitespace=false,		% sets if automatic breaks should only happen at whitespace
escapeinside={\%*}{*)},			% if you want to add a comment within your code
postbreak=\raisebox{0ex}[0ex][0ex]{\ensuremath{\color{red}\hookrightarrow\space}},
texcl=true,
inputpath=listings,                     % директория с листингами
}

\usepackage[left=2cm,right=2cm,
top=2cm,bottom=2cm,bindingoffset=0cm]{geometry}

%% Нумерация картинок по секциям
\usepackage{chngcntr}
\counterwithin{figure}{section}
\counterwithin{table}{section}

%%Точки нумерации заголовков
\usepackage{titlesec}
\titlelabel{\thetitle.\quad}
\usepackage[dotinlabels]{titletoc}

%% Оформления подписи рисунка
\addto\captionsrussian{\renewcommand{\figurename}{Рисунок}}
\captionsetup[figure]{labelsep = period}

%% Подпись таблицы
\DeclareCaptionFormat{hfillstart}{\hfill#1#2#3\par}
\captionsetup[table]{format=hfillstart,labelsep=newline,justification=centering,skip=-10pt,textfont=bf}

%% Путь к каталогу с рисунками
\graphicspath{{fig/}}
\usepackage{minted}

\begin{document}	% начало документа
	
	% Титульная страница
	\begin{titlepage}	% начало титульной страницы

	\begin{center}		% выравнивание по центру

		\large Санкт-Петербургский политехнический университет Петра Великого\\
		\large Институт компьютерных наук и технологий \\
		\large Кафедра компьютерных систем и программных технологий\\[6cm]
		% название института, затем отступ 6см
		
		\huge Телекоммуникационные технологии\\[0.5cm] % название работы, затем отступ 0,5см
		\large Отчет по лабораторной работе №6\\[0.1cm]
		\large Цифровая модуляция\\[5cm]

	\end{center}


	\begin{flushright} % выравнивание по правому краю
		\begin{minipage}{0.25\textwidth} % врезка в половину ширины текста
			\begin{flushleft} % выровнять её содержимое по левому краю

				\large\textbf{Работу выполнил:}\\
				\large Косенков М.А.\\
				\large {Группа:} 33531/2\\
				
				\large \textbf{Преподаватель:}\\
				\large Богач Н.В.

			\end{flushleft}
		\end{minipage}
	\end{flushright}
	
	\vfill % заполнить всё доступное ниже пространство

	\begin{center}
	\large Санкт-Петербург\\
	\large \the\year % вывести дату
	\end{center} % закончить выравнивание по центру

\thispagestyle{empty} % не нумеровать страницу
\end{titlepage} % конец титульной страницы

\vfill % заполнить всё доступное ниже пространство
	
	% Содержание
	\setcounter{page}{2}
	% Содержание
\renewcommand\contentsname{\centerline{Содержание}}
\tableofcontents
\newpage
	
	
	\section{Цель работы}
	изучение амплитудной модуляции/демодуляции сигнала.
	
	\section{Программа работы}
	\begin{itemize}
		\item Сгенерировать однотональный сигнал низкой частоты.
		\item Выполнить амплитудную модуляцию (АМ) сигнала для различных значений глубины модуляции M.
		\item Получить спектр модулированного сигнала.
		\item Выполнить модуляцию с подавлением несущей
		\item Выполнить однополосную модуляцию.
		\item Получить спектр
	\end{itemize}
	
	\section{Теоретическая информация}
	Амплитудная модуляция — вид модуляции, при которой изменяемым параметром несущего сигнала является его амплитуда. Амплитудно-модулированный сигнал $ u(t) $ имеет вид:
	\par
	\par
	$ u(t)=(1+MU_{m}\cos\Omega t)\cos(\omega _{0}t+\phi _{0}) $, где M - глубина модуляции
	\par
	Также в радиотехнике часто используют амплитудную модуляцию с подавлением несущей:
	\par
	$ u(t)=MU_{m}\cos(\Omega t)\cos(\omega _{0}t+\phi _{0}) $ 
	\par
	Это позволяет повысить КПД амплитудной модуляции, т.к. подавляется основная несущая частота, но остаются боковые информативные частоты.
	\par
	АМ с подавленной несущей частотой имеет преимущества по сравнению обычной АМ только в энергетическом смысле. Спектры боковых полос АМ сигнала являются зеркальным отражением друг друга, т. е. они несут одинаковую информацию. Поэтому одну из боковых полос можно удалить. Получившаяся модуляция называется однополосной (английский термин – single side band, SSB).
	\par
	$ u(t)=MU_{m}\cos(\Omega t)\cos(\omega _{0}t+\phi _{0})+U_{m}/2\sum \limits_{n=1}^{N}M_{n}(\cos(\omega_{0}+\Omega _{n})t+\phi _{0}+\Phi _{n}) $
	\par
	В зависимости от того, какая полоса передается, различают однополосный сигнал с верхней или нижней боковой полосой.
	\newpage
	\section{Ход работы}
	Данная работа выполнялась на языке Python.
	\par
	Был сгенерирован гармонический синусоидальный сигнал с частотой 20 Гц и амплитудой, равной 1. Для осуществления обычной амплитудной модуляции и для модуляции с подавлением несущей не было использовано никаких сторонних библиотек, однако для однополосной модуляции была использована функция библиотеки SciPy signal.hilbert() - преобразование Гильберта.
	\par
	\large {Листинг 1. lab4.py}
	\inputminted[
	frame=lines,
	framesep=2mm,
	baselinestretch=1.2,
	fontsize=\footnotesize,
	linenos
	]{python}{../lab4.py}
	
	\large {Результат работы}
	\begin{center}
		\large {Амплитудная модуляция}
	\end{center}
	
	\begin{figure}[H]
		\begin{center}
			\includegraphics[scale=0.7]{../am1_sig.png}
			\caption{Модулируемый сигнал, M < 1} 
		\end{center}
	\end{figure}
	
	\begin{figure}[H]
		\begin{center}
			\includegraphics[scale=0.7]{../am2_sig.png}
			\caption{Модулируемый сигнал, M > 1} 
		\end{center}
	\end{figure}

	\begin{figure}[H]
		\begin{center}
			\includegraphics[scale=0.7]{../am3_sig.png}
			\caption{Модулируемый сигнал, M = 1} 
		\end{center}
	\end{figure}

	\begin{figure}[H]
		\begin{center}
			\includegraphics[scale=0.7]{../am_spectrum.png}
			\caption{Спектр модулируемого сигнала} 
		\end{center}
	\end{figure}

	Для предотвращения загромождения отчета, на рисунке представлен спектр только модуляции с глубиной 1. Спектр модулируемого сигнала показывает, что присутствуют две боковые частоты и несущая. Боковые частоты находятся на равном расстоянии от несущей. Из всего этого можно сделать вывод, что амплитудная модуляция прошла корректно. 
	\newpage
	\begin{center}
		\large{Модуляция с подавлением несущей}
	\end{center}
	
	\begin{figure}[H]
		\begin{center}
			\includegraphics[scale=0.7]{../sup_sig.png}
			\caption{Модулируемый сигнал} 
		\end{center}
	\end{figure}
	
	\begin{figure}[H]
		\begin{center}
			\includegraphics[scale=0.7]{../sup_spectrum.png}
			\caption{Спектр модулируемого сигнала} 
		\end{center}
	\end{figure}

	На графике спектра четко видно отсутствие несущей частоты и присутствие двух боковых. Это дает понять, что модуляция прошла успешно. Мы добились того, что не приходится передавать лишнюю информацию в канале.
	
	\newpage
	\begin{center}
		\large{Однотональная модуляция}
	\end{center}
	
	\begin{figure}[H]
		\begin{center}
			\includegraphics[scale=0.7]{../single_sig.png}
			\caption{Модулируемый сигнал} 
		\end{center}
	\end{figure}
	
	\begin{figure}[H]
		\begin{center}
			\includegraphics[scale=0.7]{../single_spectrum.png}
			\caption{Спектр модулируемого сигнала} 
		\end{center}
	\end{figure}
	
	Теперь мы добились еще больших успехов - нам удалось отсечь одну из боковых частот. По сути, боковые частоты отражают друг друга зеркально, значит, несут одинаковую информацию. Поэтому, удалив одну из них, мы можем более эффективно использовать ресурсы канала.
	
	\newpage
	\section{Выводы}
	В ходе выполнения работы были исследованы различные виды модуляции, проанализировано различие между ними и преимущества одних над другими. Работа позволила понять, почему простая амплитудная модуляция используется в радиотехнике гораздо реже, чем модуляция с подавлением несущей или однотональная модуляция.
	\par
	Несмотря на то, что однотональная модуляция в некотором смысле выигрывает у модуляции с подавлением несущей и у простой амплитудной модуляции, она требует более сложной логики при самом процессе модуляции и демодуляции. Это усложнение может привести к более высокой стоимости аппаратного оборудования, а также к проблемам при его разработке и проектировании.
\end{document}